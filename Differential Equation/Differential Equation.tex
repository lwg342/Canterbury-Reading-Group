\documentclass[12pt,oneside]{book}
\usepackage{../mybook}\addbibresource{lib.bib}
%--------Hello World--------%
%=====================================%
\begin{document}
    \title{Differential Equation\\ (with a bias to dynamical systems)}\author{Canterbury Reading Group}\date{\today}\maketitle
    \tableofcontents
    \chapter{Differential Equations}
    Ref:Chua's Notes; Hirsh and Smale's Book
    \section{From Classic Mechanics to Differential Equations (Dynamical Systems)}
    \yellow{Newton's Equations,Review on solving ODE}
    
    \textit{Well understood (Hopefully)}
    \section{Linear Theory}
    \begin{equation}
    	\dot{x}=Ax
    \end{equation}
    \yellow{Definition of Vector Space; Subspace; Bases; Linear Operator; Vector V.S. coordinates; Change of coordinates; Decomposition}
    
    \red{My own question: Why introduce the notion of Operator? \\
    Initial Answer: Linear Map may be associated with different matrix given the change of the bases of the vector space.}


	Basic linear algebra knowledge like matrix properties,operations are omitted.
	
	
	\textit{Need to be more familiar and accurate with the concept of operators; Other parts done}
	\section{Linear System}
	\yellow{How do we find solutions? Technical Chapter; We may skip that if not interested}
	
    \textit{Know how to do it in practice; Need some efforts if we want to go through the decomposition and canonical forms}
    
    \section{Characterization by Behavior of the Solutions}
    \yellow{Critical Points: sink, source, saddle}
    
    \textit{Done; Nice qualitative geometric intuitions}
    
    \section{Generic Properties of Operators and Fundamental Theorems}
    
    \yellow{To be filled}
    
    \textit{Most theoretical parts of the books. Requires strong analysis foundations. Worth the effort. Little insight at the moment.}
    
    \section{Nonlinear Systems}
    
    \textit{Systems that I am studying... Hope I can make these things interesting and clear.}
    
    \subsection{Stability}
    \yellow{Linearization, Cone, Gradient system, Lyapunov function}
    \subsection{Attractors Beyond Equilibria}
    \yellow{Periodic attractors, Chaotic attractors}
    
    \section{Real Life Systems}
    
    \textit{What type of questions we can solve using the knowledge we learned so far?}
    
    \section{Nonautonomous Equations and Differentiablility of Flows}
    
    \textit{Advanced staff, a bridge to understand the flows on manifolds (Differential Geometry)}
    
    \section{Perturbation Theory and Structural Stability}
    
    \textit{Know the basic concepts. Needs to be familiar with some details and examples. Focus on the structure of the differential equations, what will happen if we omit some terms, i.e. we perturb the system structure.}
    
    \chapter{PDE}
    Ref: TBD
\end{document}