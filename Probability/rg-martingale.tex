\subsection{Martingale, Stopping Time, Martingale Transforms}

Conditional expectation is very important, it's a way to capture the dependence among random variables.

We begin with the definition of \key{filtration}. A filtration is a collection of sub-\(\sigma\)-algebra of \((\Omega, \mathcal{F}, P)\), such that 
\begin{equation*}
    \mathcal{F}_{n} \subset \mathcal{F}_{n+1}
\end{equation*}
It represents the accumulation of information over time. 

\begin{defn}[Martingales]
    An adapted stochastic process \((X_{t}, \mathcal{F}_{t})\) is a sub-martingale, if \(X_{t} \in L^{1}\) and 
    \begin{equation*}
        E\pqty{X_{t+1}\mid X_{t}} \geq X_{t} \qq{a.s.}
    \end{equation*}
    We write \(\mathbb{M}_{\uparrow}\) for the class of submartingales, \(\mathbb{M}^{s}\) for super-martingales. 
\end{defn}

We say \(H_{t}\) is a \key{predictable process}, if \(H_{t} \in \mathcal{F}_{t-1}\). And we define the \key{martingale transform} of \(H_{t}\), (similar to integration notation) as 
\begin{equation*}
    X^{t}(H_{t}) := \sum_{t=1}^{t} H_{t}(X_{t} - X_{t-1})
\end{equation*}

\subsection{Transformations that Preserves Martingale}

For \textbf{nondecreasing} convex \(\phi\), because \(E(\phi(X)) \geq \phi(E(X))\), we know \(\phi(\mathbb{M}_{\uparrow}) \in \mathbb{M}_{\uparrow}\), implicitly we require \(\phi(X_{t}) \in L^{1}\) for all \(t\). 

\subsection{Martingale Inequalities}

We can control the (expected) number of sub-martingales crossing an interval by the expectation of the end value of the sub-martingale. 
\begin{lemma}[Doob's Upcrossing Inequality]
    Let \(U_{T}(a,b)\) be the number of upcrossings of \([a,b]\) completed before time \(T\) by \(X_{t} \in \mathbb{M}_{\uparrow}\), then 
    \begin{equation*}
        E(U_{T}(a,b)) \leq (b-a)\bqty{E(X_{n} - a)_{+} - E(X_{0} - a)_{+}}
    \end{equation*}
\end{lemma}
\begin{proof}
    Consider the Buy-and-hold until upcrossing is completed strategy \(H_{t}\). Calculate the expected return. 
\end{proof}

We have a result for martingale that is similar to the \key{Kolmogorov's Maximal Inequality}. 
\begin{lemma}[\(\mathbb{M}_{\uparrow}\) Maximal Inequality]
    For \(X_{t} \in \mathbb{M}\). 
    \begin{equation*}
        P\pqty{\max_{1\leq t \leq T} X_{t} \geq \epsilon} \leq E\pqty{X_{T} \indicator{\max_{t\leq T} X_{t} \geq \epsilon }}/ \epsilon
    \end{equation*}
\end{lemma}
\begin{proof}
    Use the crossing idea that's used in proving Kolmogorov's inequality. 
\end{proof}

A corolloary is
\begin{lemma}[Doob's \(L^{p}\) inequality]
    If \(X_{t} \in \mathbb{M}\), then for \(p > 1\), 
    \begin{equation*}
        \norm{X_{T} }_{p} \leq \norm{\max_{1 \leq t\leq T} \abs{X_{t}}}_{p} \leq q \norm{X_{T}}_{p}
    \end{equation*}
\end{lemma}
\begin{proof}
    Use \(EX = \int x P(X > x) \dd{x}\), the \(\mathbb{M}_{\uparrow}\) maximal inequality for \(\abs{X_{t}}^{p}\), which is in \(\mathbb{M}_{\uparrow}\),  and Hölder. 
\end{proof}


The Buckholder's inequality relate the \(L^{p}\) norm of a martingale to the partiial sum of squared MDS. 
\begin{lemma}[Buckholder's Inequality]
    
\end{lemma}

\subsection{Martingale Convergence Theorems}
An easy way to show a sequence of random variables converges almost surely and in \(L^{p}\). 

\begin{thm}[Martingale convergence theorem]
    Let \(X_{t} \in \mathbb{M}_{\uparrow}\), then if \(\sup_{t} E\abs{X_{t}} \leq \infty\), 
\end{thm}

\begin{thm}[Martingale \(L^{p}\) Convergence]
    
\end{thm}

\key{Uniform integrability condition}. 
\begin{thm}[Martingale \(L^{1}\) convergence]
    
\end{thm}

\subsection{Martingale Law of Large Numbers}
See Hall and Hyde, talks about conditions under which \(b_{n}^{-1} X_{n} \to 0\)  in probability, a.s., or in \(L^{p}\), where \(b_{n}\) can be random norming and \(X_{n}\) can be mixingales. 

\subsection{Martingale Central Limit Theorems}

\begin{thm}
    Under the following conditions for a martingale array \(S_{ni}\), and \(X_{n}\) being the MDS. 
    \begin{enumerate}
        \item Asymptotically negligible differences: \(\max \abs{X_{i}} \to_{p} 0\)
        \item Conditional variance condition: \(\sum X_{i}^{2} \goto{p} \eta^{2}\)
        \item Bounded \(E(\max_{i} X_{ni}^{2})\). 
    \end{enumerate}
    We have (stable) convergence of martingale \(S_{n} \rightsquigarrow_{s} Z\). Where \(\rightsquigarrow \) means convergence stably. 
\end{thm}

\subsection{Applications}

\subsubsection{Mixingales}
See p.19 of Hall and Hyde, it includes uniform mixing sequence, linear process, etc. 

\subsubsection{U-statistics}

\subsubsection{Estimation from Stochastic Processes}
