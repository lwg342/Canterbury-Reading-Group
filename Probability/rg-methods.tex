This will collect the methods we used in the probability book.

\section{Property Determining Class}

\begin{method}[Measurability Determining]
    
\end{method}

\begin{method}[Monotone Class Theorem]
    
\end{method}

\begin{method}[\(\pi - \lambda\) Theorem]
    
\end{method}


\begin{method}[Independence determined by \(\pi\)-systems]
    
\end{method}

\begin{method}[Weak Convergence]
    
\end{method}

\section{Thinning and Truncating}

The purpose is to have a sequence that has better property than the original sequence. 

\begin{method}[Cauchy's Condensation Test]
    Let \((f(n))\) be a \key{decreasing} sequence of real numbers, then \(\sum f(n)\) converges iff \(\sum 2^{n} f(2^{n})\) converges.
\end{method}
\begin{proof}
    Grouping terms. 
\end{proof}

\begin{method}[Thinning in Convergence in Topological Space]\label{method:thin-topology}
    Let \(x_{n}\) be a sequence in a topological space \(\mathcal{X}\), then \(x_{n} \to x\) in that topology, iff for all subseqeunce of \(x_{n}\) there exists a further subsequence that converges to \(x\).
\end{method}
\begin{proof}
    \(\implies\) is obvious. For the other direction, suppose \(x_{n}\) is not converging, then there exists an open neighbourhood \(\mathcal{N}\) of \(x\) such that for all \(N\), there exists \(n>N\), \(x_{n} \notin \mathcal{N}\), which will constitute a subsequence that has no further subsequence that converges.
\end{proof}

\begin{method}[Thinning and Control]
    Let \(c_{n} \uparrow \infty\) be a sequence of constants, then \(X_{n} /c_{n} \to 1\) if we can find a subsequence that converges and \(c_{n+1}/c_{n} \to 1\).
\end{method}

\begin{method}[Truncating]
    Let \(Y_{n} = X_{n} \mathbf{1}(X_{n} \leq M)\), then \(Y_{n}\) and \(X_{n}\) are equivalent iff \(P\pqty{Y_{n} \neq X_{n} i.o.} = 0\). 
\end{method}


\section{Complexity Control}
    \begin{method}[Compactness]
        For any compact set \(E\) and a family of open sets covering \(E\), there exists finite subcover of \(E\). 
    \end{method}
    
    \begin{method}[Vitali's Covering]
        For any \(E \subset \mathbb{R}^{d}\) and Vitali Covering \((I_{\alpha}: \alpha \in \mathcal{A})\) of \(E\), there exists an at most countable \red{disjoint} \((I_{j}: j\in \mathbb{N})\) such that \(\lambda^{*}(E \setminus \cup I_{j}) = 0\). 
    \end{method}

    \begin{method}[See Folland Lemma 3.15, Wiener's Covering]
        
    \end{method}
    
    \begin{method}[Entropy Method]
        
    \end{method}